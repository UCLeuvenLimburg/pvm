\section{Move Constructor}

\frame{\tableofcontents[currentsection]}

\begin{frame}
  \frametitle{Move Constructor}
  \begin{itemize}
    \item Introduced in \cpp11
    \item Optimisation
    \item Called when object is ``moved''
    \item Lets an object ``steal'' internal fields from other object
    \item \texttt{T\&\&} is ``r-value reference''
          \begin{itemize}
            \item Yet another kind of pointery referency thingamajig
            \item We won't discuss the details
          \end{itemize}
  \end{itemize}
  \vskip4mm
  \code[language=c++14]{move-constructor.cpp}
\end{frame}

\begin{frame}
  \frametitle{Example}
  \code[language=c++14]{move-example.cpp}
  \begin{itemize}
    \item We're assuming no named return value optimisation
    \item When {\tt foo} returns, {\tt ns} must be copied to {\tt result}
    \item Since {\tt foo} returns, {\tt ns} must be destroyed too
    \item So, we're making a copy only to destroy the original
    \item Why copy at all?
  \end{itemize}
\end{frame}

\begin{frame}
  \frametitle{Move Constructor}
  \begin{itemize}
    \item Move constructor can be used in such circumstances
    \item One object ``takes over'' all data from the original
    \item The original object is ``emptied''
    \item Emptying original important: its destructor will still be called
  \end{itemize}
  \code[language=c++14,width=.6\linewidth]{move-constructor-syntax.cpp}
\end{frame}

\begin{frame}
  \frametitle{Example}
  \code[language=c++14,font=\small]{int-array.cpp}
\end{frame}

\begin{frame}
  \frametitle{Visualisation: Copy Constructor}
  \begin{center}
    \begin{tikzpicture}[object/.style={draw,minimum width=3cm,minimum height=.75cm,drop shadow,fill=white}]
      \visible<1-20>{
        \node[object] (v1) at (0,0) {\tt vector<char>};
      }

      \visible<1-19>{
        \draw[xstep=.25cm,ystep=.25cm,xshift=-0.125cm] (0,-1) grid (0.25,-5);
        \draw[-latex] (v1.south) -- (0,-1);
        \foreach[count=\i] \c in {a,b,c,d,e,f,g,h,i,j,k,l,m,n,o,p} {
          \tikzmath{
            real \j;
            \j = -1 - 0.25 * \i + 0.125;
          }

          \node[font=\tiny] at (0,\j) {\c};
        }
      }

      \visible<2-21>{
        \node[object] (v2) at (5,0) {\tt vector<char>};
      }

      \visible<3->{
        \draw[xstep=.25cm,ystep=.25cm,xshift=4.875cm] (0,-1) grid (0.25,-5);
        \draw[-latex] (v2.south) -- (5,-1);
      }

      \foreach[count=\i] \c in {a,b,c,d,e,f,g,h,i,j,k,l,m,n,o,p} {
        \tikzmath{
          int \slideindex;
          real \j;
          \slideindex = int(\i + 3);
          \j = -1 - 0.25 * \i + 0.125;
        }

        \visible<\slideindex>{
          \draw[-latex] (0,\j) -- (5,\j);
        }

        \visible<\slideindex->{
          \node[font=\tiny] at (5,\j) {\c};
        }
      }
    \end{tikzpicture}
  \end{center}
\end{frame}

\begin{frame}
  \frametitle{Visualisation: Move Constructor}
  \begin{center}
    \begin{tikzpicture}[object/.style={draw,minimum width=3cm,minimum height=.75cm,drop shadow,fill=white}]
      \visible<1-3>{
        \node[object] (v1) at (0,0) {\tt vector<char>};
      }

      \draw[xstep=.25cm,ystep=.25cm,xshift=-.125cm] (0,-1) grid (0.25,-5);
      \foreach[count=\i] \c in {a,b,c,d,e,f,g,h,i,j,k,l,m,n,o,p} {
        \tikzmath{
          real \j;
          \j = -1 - 0.25 * \i + 0.125;
        }

        \node[font=\tiny] at (0,\j) {\c};
      }

      \visible<1-2>{
        \draw[-latex] (v1.south) -- (0,-1);
      }

      \visible<2-4>{
        \node[object] (v2) at (5,0) {\tt vector<char>};
      }

      \visible<3>{
        \draw[-latex] (v2.south) -- (0,-1) node[midway,below,sloped] {yoink!};
      }

      \visible<4>{
        \draw[-latex] (v2.south) -- (0,-1);
      }
    \end{tikzpicture}
  \end{center}
\end{frame}

\begin{frame}
  \frametitle{Move Constructors}
  \begin{itemize}
    \item You don't need to know the details
    \item Move constructors important to understand smart pointers
  \end{itemize}
\end{frame}


%%% Local Variables:
%%% mode: latex
%%% TeX-master: "constructors"
%%% End:
